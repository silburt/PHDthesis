\documentclass[12pt,letter]{aastex}
\usepackage[hyphens]{url}
%\usepackage[breaklinks]{hyperref}   %This is the key package which allows url wrapping.
%\usepackage[draft]{hyperref}
%\usepackage[hyphenbreaks]{breakurl}
\usepackage{longtable}
\usepackage{amsmath,amssymb,multirow,dcolumn,fancyhdr,charter,graphicx}
\usepackage{graphics}
\usepackage{xspace}
\usepackage{color,ulem,epstopdf}
\newcommand{\x}{\sout}
\newcommand{\w}{\color{red}}
\newcommand{\vdag}{(v)^\dagger}
\def\memohr#1{\color{blue}$HR[${\bf #1}$]$ \color{black}}
\def\memoas#1{\color{red}$AS[${\bf #1}$]$ \color{black}}
\def\commentas#1{\color{purple}$AS[${\bf #1}$]$ \color{black}}
\def\refr#1{{\bf #1}}
\usepackage{tikz}
\def\checkmark{\tikz\fill[scale=0.4](0,.35) -- (.25,0) -- (1,.7) -- (.25,.15) -- cycle;} 

%commands
\newcommand{\reb}{{\sc \tt REBOUND}\xspace}
\newcommand{\whfast}{{\sc \tt WHFAST}\xspace}
\newcommand{\ias}{{\sc \tt IAS15}\xspace}
\newcommand{\emcee}{{\sc \tt EMCEE}\xspace}
\newcommand{\Lagr}{\mathcal{L}}
\newcommand{\kep}{{\it Kepler}\xspace}

%journal abbreviations
      
\date{Draft version: \today}

\begin{document}
\section{Introduction}

Proposed sections in the introduction:
\begin{enumerate}
\item Exoplanet detection -- Dominant detection methods = Transit, RV. Basic statistics of exoplanets discovered by Kepler Space Telescope. 
\item Planet Formation -- MMSN, Core accretion/GI models, Planetesimal Formation, Nice Model. 
\item Planet Dynamics -- Mean Motion Resonance, (planetesimal and gas) Migration,  Stability.
\item Numerical Integration -- Hamiltonian Dynamics, Close Encounters, etc. 
\end{enumerate}

Below I am going to elaborate on planet dynamics. 

\section{Planet Dynamics}
\subsection{Mean Motion Resonance (MMR)}
\label{sec:MMR}
MMR occurs when the orbital period of one planet is an integer ratio of another. 
Like other types of resonance that occur in nature, MMR results in the amplitude growth of various quantities characterizing the system, like eccentricity, semi-major axis and the longitude of pericentre \citep{SSD1999}. 
As a result, the presence of MMR can strongly affect the formation, evolution and longterm stability of planetary systems in a diversity of ways.
For example, Kirkwood gaps are unstable regions in the asteroid belt carved by MMRs with Jupiter, while Pluto and Neptune are protected from colliding due to a 3:2 MMR, even though they are on crossing orbits. 

For every $p:q$ MMR (where $p$ and $q$ are integers) there are two important resonant angles:
\begin{align*}
\begin{split}
\phi_1 &= p\lambda_1 - q\lambda_2 + \varpi_1 \\
\phi_2 &= p\lambda_1 - q\lambda_2 + \varpi_2 
\end{split}
\end{align*}
where $\lambda$ is the mean longitude and $\varpi$ is the longitude of periapse. 
For planets to be in MMR the time variation of at least one resonant argument, $\dot{\phi}$, must be zero.
%, however there are exceptions to this rule.
%For example, planets with commensurate period ratios are not required to be in MMR, and planets a few percent away from period commensurability can be found to be in MMR \citep[e.g.][]{Lee2002}. 
%In addition, circulating resonant angles can be associated with planets in MMR while librating resonant angles can be associated with planets not in MMR \citep{Delisle2012}.
%As a result, it can be difficult to constrain the formation histories of discovered planetary systems. 

\begin{figure}
\centering
\includegraphics[width=1.00\textwidth]{Figures/KeplerPeriods.png}
\caption{
\footnotesize Period ratios of Kepler planets, image from \citet{Goldreich2014}.}
\label{fig:KepMMR}
\end{figure}

The strength of a given MMR is related to its width, which in turn is related to the order of the resonance, and the magnitude of $p$ and $q$. 
In general, the lower the order and the smaller $p$ and $q$ are the stronger the resonance \citep{SSD1999}, making the 2:1 and 3:2 MMRs are the most probable trapping locations in nature. 
Figure~\ref{fig:KepMMR} shows the distribution of period ratios for planets discovered by \kep, along with the locations of first and second order MMRs. 
As can be seen, statistical excesses of planets exist near the 2:1 and 3:2 MMR \citep{Lissauer2011,Fabrycky2014,Steffen2015}, supporting the idea that these resonant locations trap planets.

However, most planets from these statistical pileups are a few percent away from exact period commensurability, and dissipative mechanisms have been proposed to transport these planets from exact MMR. 
The most popular dissipative mechanisms are tidal \citep{LithwickWu2012, Batygin2013, Delisle2014}, protoplanetary \citep{Rein2012b, Baruteau2013, Goldreich2014}, and planetesimal \citep{Moore2013, Chatterjee2015}. 
The formation implications for each mechanism are different, and no clear consensus has yet emerged.

%Structure of MMR -- pendulum model, Separatrix, libration of resonant angles. 
%With the exception of a these statistically significant excesses at first order resonances, the period distribution of planets is to first order uniform.

\subsection{Migration}
\subsubsection{Planetesimal-Driven Migration}
Planets can migrate in the presence of a massive planetesimal disk.
Planetesimals that pass through the Hill sphere of a planet will strongly interact gravitationally, exchanging angular momentum with the planet \citep{Ida2000, Kirsh2009}.
If there is an asymmetry to the number of planetesimals interacting with the planet on its near and far sides, migration will result. 
However, to guarantee migration, planetesimal orbits must decouple from the planet, which is achieved by either ejecting the planetesimals from the planetary system or by passing the planetesimals to another, neighbouring planet. 
In addition, for continuous migration the planet must constantly encounter fresh, dynamically cold planetesimals.  

It is believed that such migration occurred for Neptune, which migrated outwards into the Kuiper belt, shepherding planetesimals inwards to Jupiter which subsequently ejected them from the Solar System \citep{Fernandez1984}.
This idea is well supported by the observations of the outer Solar System, which show that Pluto along with a host of smaller bodies orbit in stable 3:2 MMRs with Neptune \citep{Malhotra1993, Malhotra1995}, a natural outcome from outward, planetesimal-driven migration. 

\subsubsection{Gas-Driven migration}
A planet embedded in a protoplanetary disk can migrate via an exchange of angular momentum from disk-planet torques \citep{Goldreich1980}.
Since the discovery of the first hot Jupiter \citep{Mayor1995}, gas-driven migration is believed to play an important role in shaping exoplanetary systems \citep{Lin1996}.
Planet migration comes in three main flavours, Type I, Type II and Type III. 

Type I migration occurs when low-mass planets are fully embedded in a protoplanetary disk and do not significantly perturb the disk structure. 
At particular resonant locations, known as ``Linblad resonances'', density waves are excited due to gravitational interactions between the planet and disk \citep{Goldreich1979}. 
These density waves exchange angular momentum with the planet, and migration occurs when the inner and outer disk interact asymmetrically with the planet \citep{Goldreich1979}.

Type II migration occurs when high-mass Jovian planets significantly modify the structure of the surrounding protoplanetary disk. 
In particular, a gap in the disk is opened up by the planet, and is locked in place by the boundaries of the gap.
The planet then migrates inwards the same rate as the local disk due to ordinary viscous evolution \citep{Armitage2010}.  

Type III. 

In comparison to planetesimal migration, gas-driven migration is still not well understood. 
In particular, standard calculations of gas-driven migration are too quick by an order of magnitude \citep{Lin1986, Tanaka2002}, causing planets to spiral into their central stars before the protoplanetary disk has dispersed.
In contrast, recent work \citep{Fung2017} suggests that planets actually do not migrate that much and tend to be better behaved than originally believed. 
A consensus on migration has yet to be established, but it is clear that some form of migration occurs in the universe due to the large number of planets in MMR.

\subsection{Stability}

\bibliographystyle{apj}
\bibliography{intro_ver1.bib}

\end{document}

