\chapter{Conclusions \& Future Work}
\label{chap:conc}
This thesis analyzed the statistics, formation and stability of exoplanetary systems. 
Five main results have been presented, and are summarized below. 

\section{Summary of Results}
\subsection{Occurrence Rate of \kep Planets}
In Chapter~\ref{chap:Stats} we used the \citet{Ramirez2014} \kep catalog to calculate the occurrence rate of small planets, accounting for detection biases and radius errors. 
We found that Kepler planets are preferentially peaked at $2-2.8R_\oplus$, with their numbers decreasing gradually toward smaller sizes, with a roughly log-uniform period distribution.
The average number of planets per star with periods between $20$ and $200$ days and radii between $1$ and $4R_\oplus$ is $0.46 \pm 0.03$. 
Such planets have likely experienced little photoevaporation, and may reflect the "primordial" planet population. 
Upon extrapolation we obtain an occurrence rate, for Earth-like planets within the "habitable zone" (as calculated by 1-D climate models), of $6.4^{+3.4}_{-1.1}\%$. 

\subsection{Tides Can't Explain Planets Close to MMR}
In Chapter~\ref{chap:Tides} we used an optimistic theoretical estimate for the minimum initial eccentricity required by \kep planets to explain the current observed spacing, and complimented these calculations with N-body simulations.
Analyzing 27 \kep systems having planets within $6\%$ of the 2:1 MMR, we found that the initial eccentricities required to explain the observed spacings are unreasonable from simple dynamical arguments.
Furthermore, our numerical simulations revealed "resonant tugging", an effect which conspires against the migration of resonant planets away from the 2:1 MMR, requiring even higher initial eccentricities in order to explain the current \kep distribution. 
In summary, we found that tides alone cannot explain planets close to 2:1 MMR, and additional mechanisms are required to explain these systems. 

\subsection{\hermes: A Hybrid Integrator}
Chapter~\ref{chap:Hermes} presented \hermes, a new hybrid integration scheme for long-term simulations of planetary systems undergoing close encounters or planetesimal-driven migration. 
Distant particles are integrated using \whfast, while close particles are integrated with \ias.
In addition, we created an adaptive routine for optimizing the close encounter boundary to help maintain accuracy whilst close encounters are occurring.
Since \whfast is symplectic, \ias is accurate to machine precision and both of them are unbiased, the energy error grows sub-linearly with time under the assumption that either impact parameters are randomly distributed or close encounters are rare.
We found that \hermes provides a good balance between speed and accuracy, neither achieved by the individual symplectic or non-symplectic integrators alone.

\subsection{Machine Learning and Planet Stability}
In Chapter~\ref{chap:Stability} we showed that characterizing the complicated and multi-dimensional stability boundary of tightly packed systems is amenable to machine learning methods. 
In particular, training a state-of-the-art machine learning algorithm on physically motivated features yields an accurate classifier of stability in packed systems. 
On the stability timescale we investigated ($10^7$ orbits), our trained machine was 3 orders of magnitude faster than direct N-body simulations. 
Optimized machine learning classifiers for dynamical stability may thus prove useful across the discipline, e.g., to characterize the exoplanet sample discovered by the upcoming Transiting Exoplanet Survey Satellite (TESS).

\subsection{Analyzing HD155358}
In Chapter~\ref{chap:HD155358} we analyzed the RV data for the star HD155358, which hosts two Jovian-sized planets near 2:1 MMR. 
Using a Bayesian model parameterized by the orbital elements of each planet, we showed that excluding planet-planet interactions can yield statistically different orbital solutions, leading to different implications for formation and stability. 
Using our updated orbital solution we calculate a high likelihood that the planets are in MMR. 

In addition, we conducted a stability analysis by drawing samples from our posterior distribution and simulating them for $10^9$ years, finding that our best-fit values land firmly in a stable region of parameter space.
We also explored a series of formation models that migrate the planets into MMR and generated synthetic RV curves to fit directly to the observed data. 
We found that a number of formation models fit the RV data surprisingly well, with some migration parameters being ruled out.

\section{Future Work and Directions}
From Chapter~\ref{chap:Tides}, a primary reason why tides cannot transport planets from exact MMR to a few percent wide is due to "Resonant Tugging" (Section~\ref{sec:restugg}).
This effect was seen using two different tidal prescriptions, and is believed to be a real effect vs. a numerical artifact. 
However, this effect was not analytically derived and is poorly understood. 
Additional work is needed to understand the physics behind this process, as well as its range of applicability.
Since the effect was seen for moderate-to-high eccentricities, a good starting point would be to expand the Resonant Hamiltonian to include higher-order terms. 

Chapter~\ref{chap:Hermes} presented \hermes, a hybrid integrator for planetesimal migration. 
The integrator was originally designed to investigate the role of planetesimals in transporting planets from MMR, however this research was never realized during my thesis. 
Most recently, \citet{Chatterjee2015} investigated this question for two Neptune mass planets and found that $0.2m_N$ (where $m_N$ is the mass of Neptune) of nearby, dynamically cold, planetesimals was required to break the resonance. 
However, this result warrants further investigation. 
For example, given the process of planetary formation (Section~\ref{sec:PF}) is it reasonable to assume that $0.2m_N$ of nearby, dynamically cold planetesimals would be available?
In addition, how would the initial migration phase into MMR excite, eject and collide nearby planetesimals?
In summary, a self-consistent model of MMR planets embedded in a planetesimal disk has yet to be made, and \hermes is perfect for the job.

Chapter~\ref{chap:Stability} showed that a machine trained on sample N-body integrations can accurately predict the longterm stability of new systems based off their initial conditions and early evolution. 
The training set consisted of equal mass planets on circular orbits and simulated for $10^7$ orbits. 
Substantial work can still be done to make this a valuable asset to the exoplanet community.
Primarily, this work can be extended to planetary systems more representative of the \kep population, with unequal mass planets on non-circular orbits simulated for an order of magnitude longer. 
In addition, for other scientists to use this tool, an easy-to-use pipeline must be created that can input a given planetary system and output a probability of longterm stability. 

Finally, in Chapter~\ref{chap:HD155358} we used a simple formation model to fit the RV curve of HD155358 and determine the initial conditions of the system. 
This simple model could be replaced with a full 3D hydrodynamical model, further constraining the initial conditions of the system. 

\memoas{Other possible things to mention:
\begin{enumerate}
\item Improve \hermes more? The ultimate integrator would be \mercury switching scheme + WHDS as the base integrator.
\item Implement sophisticated collisions into \reb a la Leinhardt \& Stewart.
\end{enumerate}
}
