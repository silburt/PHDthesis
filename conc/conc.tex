\chapter{Conclusions \& Future Work}
\label{chap:conc}
This thesis analyzes the statistics, formation and stability of exoplanetary systems. 
Five main results have been presented, and are summarized below. 

\section{Main Results}


\section{Future Work and Directions}
From Chapter~\ref{chap:Tides}, a primary reason why tides cannot transport planets from exact MMR to a few percent wide is due to "Resonant Tugging" (Section~\ref{sec:restugg}).
This effect was seen using two different tidal prescriptions, and is believed to be a real effect vs. a numerical artifact. 
However, this effect was not analytically derived and is poorly understood. 
Additional work is needed to understand the physics behind this process, as well as its range of applicability.
Since the effect was seen for moderate-to-high eccentricities, a good starting point would be to expand the Resonant Hamiltonian to include higher-order terms. 

Chapter~\ref{chap:Hermes} presented \hermes, a hybrid integrator for planetesimal migration. 
The integrator was originally designed to investigate the role of planetesimals in transporting planets from MMR, however this research was never realized during my thesis. 
Among others, \citet{Chatterjee2015} investigated this question for two Neptune mass planets and found that $0.2m_N$ (where $m_N$ is the mass of Neptune) of nearby, dynamically cold, planetesimals was required to break the resonance. 
However, this result warrants further investigation. 
For example, given the process of planetary formation (Section~\ref{sec:PF}) is it reasonable to assume that $0.2m_N$ of nearby, dynamically cold planetesimals would be available?
In addition, how would the initial migration phase into MMR further excite, eject and collide these nearby planetesimals?
In summary, a self-consistent model of MMR planets embedded in a planetesimal disk has yet to be done, and \hermes would be perfect for the job.

Chapter~\ref{chap:Stability} showed that a machine trained on sample N-body integrations can accurately predict the longterm stability of new systems based off their initial conditions and early evolution. 
The training set consisted of equal mass planets on circular orbits and a limited separation range. 
Substantial work can still be done to make this a valuable asset to the exoplanet community.
Primarily, this work can be extended to planets more representative of the \kep population, having systems with unequal masses on non-circular orbits, and simulated for an order of magnitude longer. 
In addition, for other scientists to use this tool, an easy-to-use pipeline must be created that can input a given planetary system and output a probability of longterm stability. 

Finally, in Chapter~\ref{chap:HD155358} we used a simple formation model to fit the RV curve of HD155358 and determine the initial conditions of the system. 
This simple model could be replaced with a full 3D hydrodynamical model, further constraining the initial conditions of the system. 

\begin{enumerate}
\item Further investigate Resonant Tugging.
\item Use \hermes to investigate planetesimals near MMR.
\item \memoas{Improve \hermes more? The ultimate integrator would be \mercury switching scheme + WHDS as the base integrator.}
\item \memoas{Sophisticated Collisions a la Leinhardt \& Stewart.}
\item Create a pipeline where scientists can input a planetary system into a trained machine learning model and receive an output probability of stability. 
\item Develop a more sophisticated formation model for HD155358 and fit to the RV curve. This would constrain the formation scenario of HD155358 even further.
\end{enumerate}
